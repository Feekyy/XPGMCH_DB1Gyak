\documentclass[12pt]{article}

\usepackage[magyar]{babel}

\date{}

\begin{document}
\begin{titlepage}
\begin{center}
\textbf{\Huge Jegyzőkönyv}
\vspace{0.5cm}
\\ \Large Féléves feladat
\vspace{0.25cm}
\\ \Large Gyilkosok adatbázisa
\end{center}
\vspace{13cm}
\large Készítette: Vitkolczi Dániel \\
\large Neptun: XPGMCH \\
\large Dátum: \today
\end{titlepage}
\textbf{A feladat rövid leírása: } A feladat egy adatbázis létrehozása volt, egy általunk választott témában. Az adatbázisnak minden összetevőjét el kell készíteni, ami egy relációs modellt, egy ER modellt, majd egy relációs sémát jelentett. Ezek elkészítésénél figyelembe kell venni ezek elkészítésének szabályait és/vagy követelményeit. Mindezeket egy, a gyakorlat vezető által megadott, egyik rajzprogrammal kellett elkészíteni. Aztán meg kellett mutatni, hogy az adatbázisunk működik a gyakorlatban is, így le kellett kódolni a táblákat, fel kell őket tölteni adattal, majd pár lekérdezés a biztonság kedvéért. Mindezt SQL Shell környezetben, a parancsokat pedig külön txt fájlba kell menteni. \\
Jómagam a sorozatgyilkosokat választottam témának, de ezt később kicsit ki kellett bővítenem, csupán azért, hogy az adatbázisom néhány funkciója értelmet nyerjen. Rajzprogramnak a draw.io-ot választottam. Van néhány hiányossága - nem lehet szaggatott vonalas aláhúzást tenni -, azonban könnyen és gyorsan lehet a rajzomat mozgatni, skálázni, kiexportálni különböző formátumban. Továbbá az SQL Shell mellett én egy kicsit az SQL Workbenchet is használtam, csupán az átláthatóság kedvéért, illetve a jóval gyorsabb ellenőrizhetőségért. \\
\textbf{1. feladat:} Megvalósításnál végig egy pókhálószerű ábrát képzeltem el, mivel ebben a rendszerben minden egy a személy nevét rejtő tábla fogja össze. Azonban ezt később jócskán módosítani kellett, ugyanis a fent leírt nem elégíti ki a feladat követelményeit. Ezen felül ez a modell is még módosult(mezők egybe-, és szétszedése, átcsoportosítása, illetve a kapcsolatok helyenkénti újragondolása) miközben a feladat többi pontját csináltam meg. \\
A megvalósításhoz draw.io-ot használtam. Ide tartozó fájlok: ERxpgmch.drawio, ERxpgmch.png
\\
\textbf{2. feladat:} Relációs modellről való átkonvertálás ott ütközött problémába, hogy rengeteg felesleges táblát hoztam létre, ezek egy része szimplán felesleges volt, vagy rosszul konvertáltam. Későbbiekben ezek nagy része vagy beolvasztásra került, vagy csak szimplán töröltem, mert felesleges volt. Mivel a relációs modellt sokszor kellett módosítanom, így nyilván a belőle származó relációs modell is változott azzal együtt. \\
Elkészítéshez itt is draw.io-ot használtam. Ide tartozó fájlok: RMxpgmch.drawio, RMxpgmch.png \\
\textbf{3. feladat:} A feladat megoldása egy kis problémába ütközött, ugyanis a jegyzettben erre a típusra nem volt példa, sőt említés sem. Így egy kis egyéni interneten való kutakodás után egy Debreceni Egyetemi jegyzetet alapul véve tudtam megoldani a feladatot. Összességében az ER modellhez hasonló diagramot csináltam csak a felépítése más. A kapcsolatok kevésbé vannak kiemelve. \\
Draw.io-t használtam nnek az elkészítéséhez is. Ide tartozó fájlok: SemaXPGMCH.drawio, SemaXPGMCH.png \\
\textbf{4. feladat:} A táblák létrehozása tűnt előszörre a legegyszerűbbnek az összes pont közül, aztán kiderült, hogy sok minden kicsit bonyolultabb, mint aminek látszott (például egy kapcsolótábla megcsinálása). Ennek létrehozásánál egész végig a RM modellt vettem alapnak. \\
SQL Shellben dolgoztam elsősorban, de az ellenőrzéseket SQL Workbenchben végeztem. Ide tartozó fájlok: CreateXPGMCH.txt, CreateXPGMCH.png, és a \textit{tables} mappa tartalmazza a kész táblákat. \\
\textbf{5. feladat:} Itt az elsődleges problémát az okozta, hogy várakozásaimmal ellentétben meglepően nehéz információt összeszedni gyilkosokról (például a használati eszközüket ritkán tüntetik fel, illetve a működésüknél gyakran csak országot írnak, szűkebb helyet nem). \\
SQL SHellben ezt felvinni egy rémálom volt. Ide tartozó fájlok: InsertXPGMCH, Insert1.png, Insert2.png, Insert3.png és a \textit{insert} mappa tartalmazza a táblázatokat, amikbe az adatokat gyűjtöttem. \\
A feladathoz használt források: \textbf{kizárólag Wikipédia}!
\end{document}